\pagestyle{fancy}
\chapter{Επίλυση}
\section{Διαστασιολόγηση διαμήκων οπλισμών των δοκών του 1ου ορόφου}
\noindent
Για τις δοκούς με διατομή $0.3m\times0.6m$ υπολογίζεται η επικάλυψη και το στατικό ύψος ως εξής.

\[
c_{nom} = 50 mm \rightarrow d_1 = c_{nom} + \phi_h + \phi_L/2 = 50 + 8 + 14/2 = 65 mm
\]

\noindent
και

\[
d = 600 - 65 = 535 mm
\]

\noindent
Το συνεργαζόμενο πλάτος \(b_{eff}\) στις διατομές (περιοχές) των δοκών όπου συναντάται θλίψη στο άνω μέρος προκύπτουν ως εξής.

\[
b_{eff,7} = 0.5\cdot6/10 + 0.2\cdot3 = 0.9 > 0.2\cdot0.5\cdot6 = 0.6 \Rightarrow b_{eff,8} = 0.6m
\]

\[
b_{eff,78,mid} = 0.7\cdot6/10 + 0.2\cdot3 = 1.02 > 0.2\cdot0.7\cdot6 = 0.84 \Rightarrow b_{eff,78,mid} = 0.84m
\]

\[
b_{eff,89,mid} = 0.7\cdot6/10 + 0.2\cdot3 = 1.02 > 0.2\cdot0.7\cdot6 = 0.84 \Rightarrow b_{eff,78,mid} = 0.84m
\]

\[
b_{eff,9} = 0.5\cdot6/10 + 0.2\cdot3 = 0.9 > 0.2\cdot0.5\cdot6 = 0.6 \Rightarrow b_{eff,9} = 0.6m
\]

\noindent
Για τις δοκούς και για ΚΠΜ οι κατασκευαστικοί κανόνες που ορίζονται από τους Ευρωκώδικες 2 και 8 απαιτούν τα παρακάτω, με βάση τα οποία επιλέγονται οι τοποθετούμενοι οπλισμοί. Χάριν συντομίας, οι σχετικές συγκρίσεις με τα κατασκευαστικά ελάχιστα παραλείπονται, όμως διενεργούνται σε όλους τους υπολογισμούς και τους επηρεάζουν αναλόγως.

\[
\rho_{min} = \dfrac{A_{s,min}}{bd} = 0.5\dfrac{f_{ctm}}{f_{yk}} = 0.5\cdot \dfrac{2.56}{500} = 5.12\cdot 10^{-3}
\]

\[
A_{s,min} = 308mm^2
\]

\[
A_{s,min,bottom,support} = 0.5A_{s,top,support}
\]

\[
A_{s,min,bottom,support} = 0.25A_{s,bottom,mid}
\]

\subsection{Μέλος 78}
\noindent
\underline{Δεξιά παρειά κόμβου 7}:

\[
\left.
   \begin{array}{ll}
     G+\psi_2 Q + Ε & \rightarrow M_{7,8} = -60+130 = 70 \\
     G+\psi_2 Q - Ε & \rightarrow M_{7,8} = -60-130 = -190 \\
     1.35G + 1.5Q     & \rightarrow M_{7,8} = -102
   \end{array}
\right\} \Rightarrow M_{7,8,top} = -190,\; M_{7,8,bottom} = 70
\]

\[
M_{7,8,top} = -190 \rightarrow \mu_{sd} = \dfrac{M_{sd}}{b d^2 f_{cd}} = \dfrac{190}{0.3\cdot0.535^2\cdot14.16\cdot10^3} = 0.156
\]

\[
\omega_{top} = 0.973\left( 1 - \sqrt{1 - \dfrac{2\mu_{sd}}{0.973}} \right) = 0.171
\]

\[
A_{s,top} = \omega_{top}b d \dfrac{f_{cd}}{f_{yd}} = 0.171\cdot 0.3 \cdot 0.535 \cdot \dfrac{14.16}{434.78} = 0.171\cdot 0.3 \cdot 0.535 \cdot \dfrac{14.16}{434.78} = 894mm^2
\]

\noindent
Άρα τοποθετούνται:

\[
A_{\tau} = 3\Phi20 = 942mm^2
\]

\[
M_{7,8,bottom} = 70 \rightarrow \mu_{sd} = \dfrac{M_{sd}}{b_{ef} d^2 f_{cd}} = \dfrac{70}{0.3\cdot0.535^2\cdot14.16\cdot10^3} = 0.057
\]

\[
\omega_{bottom} = 0.973\left( 1 - \sqrt{1 - \dfrac{2\mu_{sd}}{0.973}} \right) = 0.058
\]

\[
A_{s,bottom} = \omega_{bottom}b d \dfrac{f_{cd}}{f_{yd}} = 0.058\cdot 0.3 \cdot 0.535 \cdot \dfrac{14.16}{434.78} = 303mm^2
\]

\noindent
Άρα τοποθετούνται:

\[
A_{\tau} = 3\Phi16 = ............
\]

\noindent
\underline{Αριστερή παρειά κόμβου 8}:

\[
\left.
   \begin{array}{ll}
       G+\psi_2 Q + Ε & \rightarrow M_{8,7} = -100-123.5 = -223.5 \\
       G+\psi_2 Q - Ε & \rightarrow M_{8,7} = -100+123.5 = 23.5 \\
       1.35G + 1.5Q     & \rightarrow M_{8,7} = -170
   \end{array}
\right \} \Rightarrow M_{8,7,top} = -223.5,\; M_{8,7,bottom} = 23.5
\]

\noindent
\underline{Άνοιγμα}:

\bigskip

\noindent
Για φόρτιση \(1.35G + 1.5Q\) από το άθροισμα των αξονικών του εισογείου προκύπτει το φορτίο, έστω $q (KN/m)$ που αναλαμβάνουν οι δοκοί του 1ου ορόφου και το ημιάθροισμα των ροπών των στηρίξεων κάθε δοκού είναι $(170+102)/2= 136KNm$.

\[
q = (...........)/12 =  KN/m =  KN/m
\]

\[
M_{7,8,mid} = 136 - ql^2/8 =
\]

\subsection{Μέλος 89}

\noindent
\underline{Δεξιά παρειά κόμβου 8}:

\[
\left.
   \begin{array}{ll}
       G+\psi_2 Q + Ε & \rightarrow M_{8,9} = -100+123.5 = 23.5 \\
       G+\psi_2 Q - Ε & \rightarrow M_{8,9} = -100-123.5 = -223.5 \\
       1.35G + 1.5Q     & \rightarrow M_{8,9} = -170
   \end{array}
\right \} \Rightarrow M_{8,9,top} = -223.5,\; M_{8,9,bottom} = 23.5
\]

\noindent
\underline{Αριστερή παρειά κόμβου 9}:

\[
\left.
   \begin{array}{ll}
       G+\psi_2 Q + Ε & \rightarrow M_{9,8} = -60-130 = -190 \\
       G+\psi_2 Q - Ε & \rightarrow M_{9,8} = -60+130 = 70 \\
       1.35G + 1.5Q     & \rightarrow M_{9,8} = -102
   \end{array}
\right \} \Rightarrow M_{9,8,top} = -190,\; M_{9,8,bottom} = 70
\]

\[
M_{9,8,top} = -190 \rightarrow \mu_{sd} = \dfrac{M_{sd}}{b d^2 f_{cd}} = \dfrac{190}{0.3\cdot0.535^2\cdot14.16\cdot10^3} = 0.156
\]

\[
\omega_{top} = 0.973\left( 1 - \sqrt{1 - \dfrac{2\mu_{sd}}{0.973}} \right) = 0.171
\]

\[
A_{s,top} = \omega_{top}b d \dfrac{f_{cd}}{f_{yd}} = 0.171\cdot 0.3 \cdot 0.535 \cdot \dfrac{14.16}{434.78} = 0.171\cdot 0.3 \cdot 0.535 \cdot \dfrac{14.16}{434.78} = 894mm^2
\]

\[
M_{9,8,bottom} = 70 \rightarrow \mu_{sd} = \dfrac{M_{sd}}{b d^2 f_{cd}} = \dfrac{70}{0.3\cdot0.535^2\cdot14.16\cdot10^3} = 0.057
\]

\[
\omega_{bottom} = 0.973\left( 1 - \sqrt{1 - \dfrac{2\mu_{sd}}{0.973}} \right) = 0.058
\]

\[
A_{s,bottom} = \omega_{bottom}b d \dfrac{f_{cd}}{f_{yd}} = 0.058\cdot 0.3 \cdot 0.535 \cdot \dfrac{14.16}{434.78} = 303mm^2
\]

\noindent
\underline{Άνοιγμα}:

\bigskip

\noindent
Για φόρτιση \(1.35G + 1.5Q\)

\[
M_{8,9,mid} =
\]
