\pagestyle{fancy}
\chapter{Επίλυση}
\section{Διαστασιολόγηση διαμήκων οπλισμών των δοκών του 1ου ορόφου}
\noindent
Για τις δοκούς με διατομή $0.3m\times0.6m$ υπολογίζεται η επικάλυψη και το στατικό ύψος ως εξής.

\[
c_{nom} = 50 mm \rightarrow d_1 = c_{nom} + \phi_h + \phi_L/2 = 50 + 8 + 14/2 = 65 mm
\]

\noindent
και

\[
d = 600 - 65 = 535 mm
\]

\noindent
Το συνεργαζόμενο πλάτος \(b_{eff}\) στις διατομές (περιοχές) των δοκών όπου συναντάται θλίψη στο άνω μέρος προκύπτουν ως εξής.

\[
b_{eff,7} = 0.5\cdot6/10 + 0.2\cdot3 = 0.9 > 0.6 \Rightarrow b_{eff,8} = 0.6\cdot2+0.3 = 1.5m
\]

\[
b_{eff,78,mid} = 0.7\cdot6/10 + 0.2\cdot3 = 1.02 > 0.84 \Rightarrow b_{eff,78,mid} = 0.84\cdot2+0.3 = 1.98m
\]

\[
b_{eff,8} = 0.3\cdot6/10 + 0.2\cdot3 = 0.78 > 0.36 \Rightarrow b_{eff,8} = 0.36\cdot2+0.3 = 1.02m
\]

\[
b_{eff,89,mid} = 0.7\cdot6/10 + 0.2\cdot3 = 1.02 > 0.84 \Rightarrow b_{eff,78,mid} = 0.84\cdot2+0.3 = 1.98m
\]

\[
b_{eff,9} = 0.5\cdot6/10 + 0.2\cdot3 = 0.9 > 0.6 \Rightarrow b_{eff,9} = 0.6\cdot2+0.3 = 1.5m
\]

\noindent
Για τις δοκούς και για ΚΠΜ οι κατασκευαστικοί κανόνες που ορίζονται από τους Ευρωκώδικες 2 και 8 απαιτούν τα παρακάτω, με βάση τα οποία επιλέγονται οι τοποθετούμενοι οπλισμοί. Χάριν συντομίας, οι σχετικές συγκρίσεις με τα κατασκευαστικά ελάχιστα παραλείπονται, όμως διενεργούνται σε όλους τους υπολογισμούς και τους επηρεάζουν αναλόγως.

\[
\rho_{min} = \dfrac{A_{s,min}}{bd} = 0.5\dfrac{f_{ctm}}{f_{yk}} = 0.5\cdot \dfrac{2.56}{500} = 5.12\cdot 10^{-3}
\]

\[
A_s = 308mm^2
\]

\[
A_{s,bottom,support} \geq 0.5A_{s,top,support}
\]

\[
A_{s,bottom,support} \geq 0.25A_{s,bottom,mid}
\]

\noindent
Για φόρτιση \(1.35G + 1.5Q\), από τη διαφορά του αθροίσματος των αξονικών του εισογείου και του 1ου ορόφου διαιρεμένη με $12m$ προκύπτει το φορτίο, έστω $q (KN/m)$, που αναλαμβάνουν οι δοκοί του 1ου ορόφου. Το ημιάθροισμα των ροπών των στηρίξεων κάθε δοκού είναι $-(170+102)/2= -136KNm$.

\[
q = \dfrac{\left(714+1785+714\right)-\left(510+1190+510\right)}{12} = 84 KN/m
\]

\[
M_{7,8,mid} = M_{8,9,mid} = - 136 + ql^2/8 = -136 + 84\cdot6^2/8 = 242 KNm
\]

\subsection{Μέλος 78}
\noindent
\underline{Δεξιά παρειά κόμβου 7}:

\[
\left.
   \begin{array}{ll}
     G+\psi_2 Q + Ε & \rightarrow M_{7,8} = -60+130 = 70 \\
     G+\psi_2 Q - Ε & \rightarrow M_{7,8} = -60-130 = -190 \\
     1.35G + 1.5Q     & \rightarrow M_{7,8} = -102
   \end{array}
\right\} \Rightarrow M_{7,8,top} = -190,\; M_{7,8,bottom} = 70
\]

\[
M_{7,8,top} = -190 \rightarrow \mu_{sd} = \dfrac{M_{sd}}{b d^2 f_{cd}} = \dfrac{190}{0.3\cdot0.535^2\cdot14.16\cdot10^3} = 0.156
\]

\[
\omega_{top} = 0.973\left( 1 - \sqrt{1 - \dfrac{2\mu_{sd}}{0.973}} \right) = 0.171
\]

\[
A_{s,top} = \omega_{top}b d \dfrac{f_{cd}}{f_{yd}} = 0.171\cdot 0.3 \cdot 0.535 \cdot \dfrac{14.16}{434.78} = 0.171\cdot 0.3 \cdot 0.535 \cdot \dfrac{14.16}{434.78} = 894mm^2
\]

\noindent
Άρα τοποθετούνται:

\[
A_{\tau} = 3\Phi20 = 942mm^2
\]

\[
M_{7,8,bottom} = 70 \rightarrow \mu_{sd} = \dfrac{M_{sd}}{b_{eff} d^2 f_{cd}} = \dfrac{70}{1.5\cdot0.535^2\cdot14.16\cdot10^3} = 0.012
\]

\[
\omega_{bottom} = 0.973\left( 1 - \sqrt{1 - \dfrac{2\mu_{sd}}{0.973}} \right) = 0.012
\]

\[
A_{s,bottom} = \omega_{bottom}b_{eff} d \dfrac{f_{cd}}{f_{yd}} = 0.012\cdot 1.5 \cdot 0.535 \cdot \dfrac{14.16}{434.78} = 303mm^2
\]

\noindent
Άρα τοποθετούνται:

\[
A_{\tau} = 2\Phi20 = 628mm^2
\]

\noindent
\underline{Αριστερή παρειά κόμβου 8}:

\[
\left.
   \begin{array}{ll}
       G+\psi_2 Q + Ε & \rightarrow M_{8,7} = -100-123.5 = -223.5 \\
       G+\psi_2 Q - Ε & \rightarrow M_{8,7} = -100+123.5 = 23.5 \\
       1.35G + 1.5Q     & \rightarrow M_{8,7} = -170
   \end{array}
\right \} \Rightarrow M_{8,7,top} = -223.5,\; M_{8,7,bottom} = 23.5
\]

\[
M_{8,7,top} = -223.5 \rightarrow \mu_{sd} = \dfrac{M_{sd}}{b d^2 f_{cd}} = \dfrac{223.5}{0.3\cdot0.535^2\cdot14.16\cdot10^3} = 0.184
\]

\[
\omega_{top} = 0.973\left( 1 - \sqrt{1 - \dfrac{2\mu_{sd}}{0.973}} \right) = 0.206
\]

\[
A_{s,top} = \omega_{top}b d \dfrac{f_{cd}}{f_{yd}} = 0.206\cdot 0.3 \cdot 0.535 \cdot \dfrac{14.16}{434.78} = 0.206\cdot 0.3 \cdot 0.535 \cdot \dfrac{14.16}{434.78} = 1074mm^2
\]

\noindent
Άρα τοποθετούνται:

\[
A_{\tau} = 4\Phi20 = 1256mm^2
\]

\[
M_{8,7,bottom} = 23.5 \rightarrow \mu_{sd} = \dfrac{M_{sd}}{b_{eff} d^2 f_{cd}} = \dfrac{23.5}{1.02\cdot0.535^2\cdot14.16\cdot10^3} = 0.006
\]

\[
\omega_{bottom} = 0.973\left( 1 - \sqrt{1 - \dfrac{2\mu_{sd}}{0.973}} \right) = 0.006
\]

\[
A_{s,bottom} = \omega_{bottom}b_{eff} d \dfrac{f_{cd}}{f_{yd}} = 0.006\cdot 1.02 \cdot 0.535 \cdot \dfrac{14.16}{434.78} = 102mm^2
\]

\noindent
Άρα τοποθετούνται:

\[
A_{\tau} = 2\Phi20 = 628mm^2
\]

\noindent
\underline{Άνοιγμα}:

\bigskip

\[
M_{7,8,mid} = 242 \rightarrow \mu_{sd} = \dfrac{M_{sd}}{b_{eff} d^2 f_{cd}} = \dfrac{242}{1.98\cdot0.535^2\cdot14.16\cdot10^3} = 0.030
\]

\[
\omega_{mid} = 0.973\left( 1 - \sqrt{1 - \dfrac{2\mu_{sd}}{0.973}} \right) = 0.031
\]

\[
A_{s,mid} = \omega_{mid}b_{eff} d \dfrac{f_{cd}}{f_{yd}} = 0.031\cdot 1.98 \cdot 0.535 \cdot \dfrac{14.16}{434.78} = 0.031\cdot 1.98 \cdot 0.535 \cdot \dfrac{14.16}{434.78} = 1057 mm^2
\]

\noindent
Άρα τοποθετούνται:

\[
A_{\tau} = 4\Phi20 = 1256 mm^2
\]

\subsection{Μέλος 89}

\noindent
\underline{Δεξιά παρειά κόμβου 8}:

\[
\left.
   \begin{array}{ll}
       G+\psi_2 Q + Ε & \rightarrow M_{8,9} = -100+123.5 = 23.5 \\
       G+\psi_2 Q - Ε & \rightarrow M_{8,9} = -100-123.5 = -223.5 \\
       1.35G + 1.5Q     & \rightarrow M_{8,9} = -170
   \end{array}
\right \} \Rightarrow M_{8,9,top} = -223.5,\; M_{8,9,bottom} = 23.5
\]

\[
M_{8,9,top} = -223.5 \rightarrow \mu_{sd} = \dfrac{M_{sd}}{b d^2 f_{cd}} = \dfrac{223.5}{0.3\cdot0.535^2\cdot14.16\cdot10^3} = 0.184
\]

\[
\omega_{top} = 0.973\left( 1 - \sqrt{1 - \dfrac{2\mu_{sd}}{0.973}} \right) = 0.206
\]

\[
A_{s,top} = \omega_{top}b d \dfrac{f_{cd}}{f_{yd}} = 0.206\cdot 0.3 \cdot 0.535 \cdot \dfrac{14.16}{434.78} = 0.206\cdot 0.3 \cdot 0.535 \cdot \dfrac{14.16}{434.78} = 1074mm^2
\]

\noindent
Άρα τοποθετούνται:

\[
A_{\tau} = 4\Phi20 = 1256mm^2
\]

\[
M_{8,9,bottom} = 23.5 \rightarrow \mu_{sd} = \dfrac{M_{sd}}{b_{eff} d^2 f_{cd}} = \dfrac{23.5}{1.02\cdot0.535^2\cdot14.16\cdot10^3} = 0.006
\]

\[
\omega_{bottom} = 0.973\left( 1 - \sqrt{1 - \dfrac{2\mu_{sd}}{0.973}} \right) = 0.006
\]

\[
A_{s,bottom} = \omega_{bottom}b_{eff} d \dfrac{f_{cd}}{f_{yd}} = 0.006\cdot 1.02 \cdot 0.535 \cdot \dfrac{14.16}{434.78} = 102mm^2
\]

\noindent
Άρα τοποθετούνται:

\[
A_{\tau} = 2\Phi20 = 628mm^2
\]


\noindent
\underline{Αριστερή παρειά κόμβου 9}:

\[
\left.
   \begin{array}{ll}
       G+\psi_2 Q + Ε & \rightarrow M_{9,8} = -60-130 = -190 \\
       G+\psi_2 Q - Ε & \rightarrow M_{9,8} = -60+130 = 70 \\
       1.35G + 1.5Q     & \rightarrow M_{9,8} = -102
   \end{array}
\right \} \Rightarrow M_{9,8,top} = -190,\; M_{9,8,bottom} = 70
\]

\[
M_{9,8,top} = -190 \rightarrow \mu_{sd} = \dfrac{M_{sd}}{b d^2 f_{cd}} = \dfrac{190}{0.3\cdot0.535^2\cdot14.16\cdot10^3} = 0.156
\]

\[
\omega_{top} = 0.973\left( 1 - \sqrt{1 - \dfrac{2\mu_{sd}}{0.973}} \right) = 0.171
\]

\[
A_{s,top} = \omega_{top}b d \dfrac{f_{cd}}{f_{yd}} = 0.171\cdot 0.3 \cdot 0.535 \cdot \dfrac{14.16}{434.78} = 0.171\cdot 0.3 \cdot 0.535 \cdot \dfrac{14.16}{434.78} = 894mm^2
\]

\noindent
Άρα τοποθετούνται:

\[
A_{\tau} = 3\Phi20 = 942mm^2
\]

\[
M_{9,8,bottom} = 70 \rightarrow \mu_{sd} = \dfrac{M_{sd}}{b_{eff} d^2 f_{cd}} = \dfrac{70}{1.5\cdot0.535^2\cdot14.16\cdot10^3} = 0.012
\]

\[
\omega_{bottom} = 0.973\left( 1 - \sqrt{1 - \dfrac{2\mu_{sd}}{0.973}} \right) = 0.012
\]

\[
A_{s,bottom} = \omega_{bottom}b_{eff} d \dfrac{f_{cd}}{f_{yd}} = 0.012\cdot 1.5 \cdot 0.535 \cdot \dfrac{14.16}{434.78} = 303mm^2
\]

\noindent
Άρα τοποθετούνται:

\[
A_{\tau} = 2\Phi20 = 628mm^2
\]

\noindent
\underline{Άνοιγμα}:

\bigskip

\[
M_{9,8,mid} = 242 \rightarrow \mu_{sd} = \dfrac{M_{sd}}{b_{eff} d^2 f_{cd}} = \dfrac{242}{1.98\cdot0.535^2\cdot14.16\cdot10^3} = 0.030
\]

\[
\omega_{mid} = 0.973\left( 1 - \sqrt{1 - \dfrac{2\mu_{sd}}{0.973}} \right) = 0.031
\]

\[
A_{s,mid} = \omega_{mid}b_{eff} d \dfrac{f_{cd}}{f_{yd}} = 0.031\cdot 1.98 \cdot 0.535 \cdot \dfrac{14.16}{434.78} = 0.031\cdot 1.98 \cdot 0.535 \cdot \dfrac{14.16}{434.78} = 1057 mm^2
\]

\noindent
Άρα τοποθετούνται:

\[
A_{\tau} = 4\Phi20 = 1256 mm^2
\]

\bigskip

\noindent\textbf{\textcolor{mygreen}{Άρα τελικά οι διαµήκεις οπλισµοί των δοκών του 1ου ορόφου, όπως προέκυψαν από τους άνωθεν υπολογισμούς, φαίνονται συγκεντρωτικά στον επόμενο πίνακα.}}

\bigskip

\begin{table}[h]
\centering\footnotesize
\begin{tabular}{| c || c | c | c | c | c |}
\hline
Θέση & Κόμβος 7 & Άνοιγμα 78 & Κόμβος 8 & Άνοιγμα 89 & Κόμβος 9 \\
\hline
\hline
Κάτω & $2\Phi20$ & $4\Phi20$ & $2\Phi20$ & $4\Phi20$ & $2\Phi20$ \\
\hline
Πάνω & $3\Phi20$ & $-$ & $4\Phi20$ & $-$ & $3\Phi20$ \\
\hline
\end{tabular}
\caption{Διαμήκεις οπλισμοί δοκών 1ου ορόφου}
\label{tab:beamrebar}
\end{table}

\section{Διαστασιολόγηση κατακόρυφων οπλισμών των υποστυλωμάτων του 1ου ορόφου σύμφωνα με τον ικανοτικό σχεδιασμό σε κάμψη}

\subsection{Υπολογισμός ροπών αντοχής δοκών στις στηρίξεις}
\noindent
Οι οπλισμοί ($A_{s1}$, $A_{s2}$) των δοκών έχουν προκύψει από τη διαστασιολόγησή τους που έγινε στην προηγούμενη ενότητα και χρησιμοποιούνται παρακάτω για τον υπολογισμό των ροπών αντοχής.

\noindent
Για την εύρεση της ροπής αντοχής των δοκών στις στηρίξεις όταν εφελκύεται το κάτω πέλμα χρησιμοποιούνται οι ακόλουθες εξισώσεις.

\[
A_c = bh - A_{s1} \rightarrow \omega_1 = \dfrac{A_{s1}}{A_c}\dfrac{f_{yd}}{f_{cd}}
\]

\[
M_{Rb} = A_{s1}f_{yd}(d-d_2)
\]

\noindent
Για την εύρεση της ροπής αντοχής των δοκών στις στηρίξεις όταν εφελκύεται το πάνω πέλμα χρησιμοποιούνται οι ακόλουθες εξισώσεις.

\[
A_c = (0.15b_{eff} + 0.45b ) - (A_{s1}+A_{s2}) \rightarrow \omega_1 = \dfrac{A_{s1}}{A_c}\dfrac{f_{yd}}{f_{cd}} \rightarrow \omega_2 = \dfrac{A_{s2}}{A_c}\dfrac{f_{yd}}{f_{cd}}
\]

\[
M_{Rb} = A_{s2}f_{yd}(d-d_2)+(\omega_1 - \omega_2)(1-0.514(\omega_1 - \omega_2))b_{eff}d^2f_{cd}
\]

\bigskip
\noindent
Τα αποτελέσματα των υπολογισμών των ροπών αντοχής (σε $KNm$) στις στηρίξεις των δοκών φαίνονται στον επόμενο πίνακα (Πίν. \ref{tab:mrdbeam}).

\begin{table}[h]
\centering\footnotesize
\begin{tabular}{| c || c | c | c |}
\hline
$M_{Rb}$ & Κόμβος 7 & Κόμβος 8 & Κόμβος 9 \\
\hline
\hline
Εφελκυσμός κάτω & $128.3$ & $128.3$ & $128.3$ \\
\hline
Εφελκυσμός πάνω & $289.6$ & $397.3$ & $289.6$ \\
\hline
\end{tabular}
\caption{Ροπές αντοχής δοκών 1ου ορόφου}
\label{tab:mrdbeam}
\end{table}

\bigskip

\subsection{Ικανοτικός σχεδιασμός υποστυλωμάτων}
\noindent
\underline{Κόμβος 7}

\bigskip

\noindent
Για φόρτιση $G+\psi_2Q+E$ έχουμε:

\[
M_{CD}=\gamma_{Rd}M_{Rb} = 1.3\cdot128.3\cdot\dfrac{37}{59.5} =103.7 KNm 
\]

\[
N_d = \dfrac{N_{above}+N_{below}}{2}=\dfrac{240+300}{2}=270 KN
\]

\[
\mu_d = \dfrac{M_d}{bh^2f_{cd}}=\dfrac{103.7}{0.35\cdot0.35^2\cdot14.16\cdot1000}=0.171
\]

\[
\nu_d = \dfrac{N_d}{bhf_{cd}}=\dfrac{270}{0.35\cdot0.35\cdot14.16\cdot1000}=0.155
\]

\[
d_1/h = 0.185 \rightarrow \omega_{tot}= 0.35
\]

\noindent
Για φόρτιση $G+\psi_2Q-E$ έχουμε:

\[
M_{CD}=\gamma_{Rd}M_{Rb} = 1.3\cdot289.6\cdot\dfrac{93}{187.5} =186.7 KNm 
\]

\[
N_d = \dfrac{N_{above}+N_{below}}{2}=\dfrac{360+540}{2}=450 KN
\]

\[
\mu_d = \dfrac{M_d}{bh^2f_{cd}}=\dfrac{186.7}{0.35\cdot0.35^2\cdot14.16\cdot1000}=0.307
\]

\[
\nu_d = \dfrac{N_d}{bhf_{cd}}=\dfrac{450}{0.35\cdot0.35\cdot14.16\cdot1000}=0.259
\]

\[
d_1/h = 0.185 \rightarrow \omega_{tot}= 0.62
\]

\noindent
Άρα για τη διαστασιολόγηση έχουμε:

\[
A_{s,tot}=\omega_{tot} b h \dfrac{f_{cd}}{f_{yd}} = 2473 mm^2
\]

\[
A_{s,min}=0.01b h = 1225 mm^2 \leq A_{s,tot}
\]

\noindent
Τοποθετούνται $8\Phi20 = 2512 mm^2$

\bigskip

\noindent
\underline{Κόμβος 8}

\bigskip

\noindent
Για φόρτιση $G+\psi_2Q+E$ έχουμε:

\[
M_{CD}=\gamma_{Rd}M_{Rb} = 1.3\cdot(397.3+128.3)\cdot\dfrac{99}{209} =323.7 KNm 
\]

\[
N_d = \dfrac{N_{above}+N_{below}}{2}=\dfrac{700+1050}{2}=875 KN
\]

\[
\mu_d = \dfrac{M_d}{bh^2f_{cd}}=\dfrac{323.7}{0.40\cdot0.40^2\cdot14.16\cdot1000}=0.357
\]

\[
\nu_d = \dfrac{N_d}{bhf_{cd}}=\dfrac{875}{0.40\cdot0.40\cdot14.16\cdot1000}=0.386
\]

\[
d_1/h = 0.185 \rightarrow \omega_{tot}= 0.70
\]

\noindent
Για φόρτιση $G+\psi_2Q-E$ έχουμε:

\[
M_{CD}=\gamma_{Rd}M_{Rb} = 1.3\cdot(397.3+128.3)\cdot\dfrac{99}{209} =323.7 KNm 
\]

\[
N_d = \dfrac{N_{above}+N_{below}}{2}=\dfrac{700+1050}{2}=875 KN
\]

\[
\mu_d = \dfrac{M_d}{bh^2f_{cd}}=\dfrac{323.7}{0.40\cdot0.40^2\cdot14.16\cdot1000}=0.357
\]

\[
\nu_d = \dfrac{N_d}{bhf_{cd}}=\dfrac{875}{0.40\cdot0.40\cdot14.16\cdot1000}=0.386
\]

\[
d_1/h = 0.185 \rightarrow \omega_{tot}= 0.70
\]

\noindent
Άρα για τη διαστασιολόγηση έχουμε:

\[
A_{s,tot}=\omega_{tot} b h \dfrac{f_{cd}}{f_{yd}} = 3647 mm^2
\]

\[
A_{s,min}=0.01b h = 1600 mm^2 \leq A_{s,tot}
\]

\noindent
Τοποθετούνται $12\Phi20 = 3768 mm^2$

\bigskip

\noindent
\underline{Κόμβος 9}

\bigskip

\noindent
Λόγω συμμετρίας, προκύπτουν ακριβώς οι ίδιες απαιτήσεις οπλισμού με τον κόμβο 7.

\bigskip

\noindent\textbf{\textcolor{mygreen}{Άρα στα ακραία υποστυλώματα τοποθετούνται 8Φ20 και στο μεσαίο 12Φ20. Να σημειωθεί πως ο οπλισμός του κεντρικού υποστυλώματος, αλλά ενδεχομένως και των ακραίων είναι αρκετά μεγάλος για τη διατομή τους (μεγάλο ω), το οποίο αν δεν οφείλεται σε υπολογιστικό λάθος, ίσως οφείλεται στα δεδομένα του προβλήματος.}}

\section{Υπολογισμός τεμνουσών σχεδιασμού των δοκών και υποστυλωμάτων του 1ου ορόφου σύμφωνα με τον ικανοτικό σχεδιασμό σε διάτμηση}

\subsection{Ικανοτική τέμνουσα δοκών}
\noindent
\underline{Μέλος 78}

\bigskip

\noindent
Αφού έχει προηγηθεί ικανοτικός σχεδιασμός υποστυλωμάτων σε κάμψη, χρειάζεται να υπολογιστεί το άθροισμα τέμνουσας λόγω σεισμού και αυτής λόγω οιονεί μόνιμων κατακορύφων φορτίων αμφιερείστου, χωρίς κάποια διόρθωση.

\bigskip

\noindent
Η μέγιστη τέμνoυσα πoυ μπoρεί να αναπτυχθεί στη δoκό λόγω σεισμoύ (+E, -E) είναι:

\[
{V^{+}}_{CD,E} = \gamma_{Rd}\dfrac{{M^{-}}_{Rd,\alpha} + {M^{+}}_{Rd,\tau}}{l_n} = 1.0\dfrac{289.6 + 128.3}{6} = 69.65 KN
\]

\[
{V^{-}}_{CD,E} = \gamma_{Rd}\dfrac{{M^{+}}_{Rd,\alpha} + {M^{-}}_{Rd,\tau}}{l_n} = 1.0\dfrac{128.3 + 397.3}{6} = 87.60 KN
\]

\noindent
Οι τέμνουσες στα άκρα της δοκού λόγω $G+\psi_2Q$ εκτιμώνται βάσει των ροπών στα άκρα, όπως προέκυψαν από την ανάλυση. Δηλαδή είναι:

\[
V_7 = \dfrac{ql}{2} + \dfrac{M_8 - Μ_7}{l} = 245.3 KN
\]

\[
V_8 = V_7 - ql = -258.7 KN
\]

\noindent
Οι τέμνουσες λόγω οιονεί μόνιμων κατακορύφων φορτίων αμφιερείστου είναι:

\[
V_{G+\psi_2Q,\alpha\mu\phi}(7) = V_{G+\psi_2Q}(7) - \dfrac{M_{G+\psi_2Q,\tau}- M_{G+\psi_2Q,\alpha}}{l_n} = 245.3 - \dfrac{-100+60}{6}= 252KN
\]

\[
V_{G+\psi_2Q,\alpha\mu\phi}(8) = V_{G+\psi_2Q}(8) - \dfrac{M_{G+\psi_2Q,\tau}- M_{G+\psi_2Q,\alpha}}{l_n} = -258.7 - \dfrac{-100+60}{6}= -252KN
\]

\noindent
Οι μέγιστες θετικές και αρνητικές τέμνουσες που είναι δυνατόν να αναπτυχθούν σε μία διατομή της δοκού (εν προκειμένω στα άκρα της), σύμφωνα με τον ικανοτικό σχεδιασμό σε διάτμηση είναι:

\[
max{V^{+}}_{Ed} (7) = {V^{+}}_{CD,E} + V_{G+\psi_2Q,\alpha\mu\phi}(7) = 69.65 + 252 = 321.65 KN
\]

\[
min{V^{-}}_{Ed} (7) = -{V^{-}}_{CD,E} + V_{G+\psi_2Q,\alpha\mu\phi}(7) = -87.60 + 252 = 164.4 KN
\]

\[
max{V^{+}}_{Ed} (8) = {V^{+}}_{CD,E} + V_{G+\psi_2Q,\alpha\mu\phi}(8) = 69.65 - 252 = -182.35 KN
\]

\[
min{V^{-}}_{Ed} (8) = -{V^{-}}_{CD,E} + V_{G+\psi_2Q,\alpha\mu\phi}(8) = -87.60 - 252 = -339.6 KN
\]

\bigskip

\noindent\textbf{\textcolor{mygreen}{Άρα για τη στήριξη στον κόμβο 7 (αρχή) κρίσιμη είναι η τέμνουσα 321.65 KN και για τη στήριξη στον κόμβο 8 (τέλος) κρίσιμη είναι η τέμνουσα 339.6 KN.}}

\bigskip

\noindent
\underline{Μέλος 89}

\bigskip

\noindent
Για το μέλος 89 ισχύουν ακριβώς τα ίδια αφού είναι απόλυτα συμμετρικό ως προς τον κόμβο 8. Οπότε τα αποτελέσματα που αφορούν τον κόμβο 7, αφορούν και τον κόμβο 9 και φυσικά τα αποτελέσματα του κόμβου 8 (κοινός κόμβος) είναι ταυτόσημα.

\subsection{Ικανοτική τέμνουσα υποστυλωμάτων}

