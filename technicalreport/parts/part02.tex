\pagestyle{fancy}
\chapter{Επίλυση}
\section{Διαστασιολόγηση διαμήκων οπλισμών των δοκών του 1ου ορόφου}
Για τις δοκούς με διατομή $0.3m\times0.6m$ εκτιμάται η επικάλυψη και το στατικό ύψος ως εξής.

\[
c_{nom} = 50 mm \rightarrow d_1 = c_{nom} + \phi_h + \phi_L/2 = 50 + 8 + 14/2 = 65 mm
\]

\noindent
και

\[
d = 600 - 65 = 535 mm
\]

\subsection{Μέλος 78}
\noindent
\underline{Δεξιά παρειά κόμβου 7}:
\[
\left.
   \begin{array}{ll}
     G+\psi_2 Q + Ε & \rightarrow M_{7,8} = -60+130 = 70 \\
     G+\psi_2 Q - Ε & \rightarrow M_{7,8} = -60-130 = -190 \\
     1.35G + 1.5Q     & \rightarrow M_{7,8} = -102
   \end{array}
\right\} \Rightarrow M_{7,8,\pi} = -190,\; M_{7,8,\kappa} = 70
\]

\[
M_{7,8,\pi} = 190 \rightarrow \mu_{sd} = \dfrac{M_{sd}}{b d^2 f_{cd}} = \dfrac{190}{0.3\cdot0.535^2\cdot14.16\cdot10^3} = 0.156
\]

\[
\omega_{\pi} = 0.973\left( 1 - \sqrt{1 - \dfrac{2\mu_{sd}}{0.973}} \right) = 0.171
\]

\[
M_{7,8,\kappa} = 70 \rightarrow \mu_{sd} = \dfrac{M_{sd}}{b d^2 f_{cd}} = \dfrac{70}{0.3\cdot0.535^2\cdot14.16\cdot10^3} = 0.057
\]

\[
\omega_{\kappa} = 0.973\left( 1 - \sqrt{1 - \dfrac{2\mu_{sd}}{0.973}} \right) = 0.058
\]

\noindent
\underline{Αριστερή παρειά κόμβου 8}:
\[
\left.
   \begin{array}{ll}
       G+\psi_2 Q + Ε & \rightarrow M_{8,7} = -100-123.5 = -223.5 \\
       G+\psi_2 Q - Ε & \rightarrow M_{8,7} = -100+123.5 = 23.5 \\
       1.35G + 1.5Q     & \rightarrow M_{8,7} = -170
   \end{array}
\right \} \Rightarrow M_{8,7,\pi} = -223.5,\; M_{8,7,\kappa} = 23.5
\]
\noindent
\underline{Άνοιγμα}:

\medskip
\noindent
Για φόρτιση \(1.35G + 1.5Q\) οι ροπές των στηρίξεων είναι \(M_{7,8} = -102\) και \(M_{8,7} = -170\). Προσεγγιστικά για το μέσο του ανοίγματος θα είναι.
\[
ql^2/12 = \left(M_{7,8} + M_{8,7}\right)/2 = \left(102 + 170\right)/2 = 136 \Rightarrow ql^2 = 1632
\]

\noindent
και

\[
M_{7,8,mid} = ql^2/24 = 1632/24 = 68 KNm
\]

\subsection{Μέλος 89}
\noindent
\underline{Δεξιά παρειά κόμβου 8}:
\[
\left.
   \begin{array}{ll}
       G+\psi_2 Q + Ε & \rightarrow M_{8,9} = -100+123.5 = 23.5 \\
       G+\psi_2 Q - Ε & \rightarrow M_{8,9} = -100-123.5 = -223.5 \\
       1.35G + 1.5Q     & \rightarrow M_{8,9} = -170
   \end{array}
\right \} \Rightarrow M_{8,9,\pi} = -223.5,\; M_{8,9,\kappa} = 23.5
\]
\noindent
\underline{Αριστερή παρειά κόμβου 9}:
\[
\left.
   \begin{array}{ll}
       G+\psi_2 Q + Ε & \rightarrow M_{9,8} = -60-130 = -190 \\
       G+\psi_2 Q - Ε & \rightarrow M_{9,8} = -60+130 = 70 \\
       1.35G + 1.5Q     & \rightarrow M_{9,8} = -102
   \end{array}
\right \} \Rightarrow M_{9,8,\pi} = -190,\; M_{9,8,\kappa} = 70
\]
\noindent
\underline{Άνοιγμα}:

\medskip
\noindent
Για φόρτιση \(1.35G + 1.5Q\) οι ροπές των στηρίξεων είναι \(M_{8,9} = -170\) και \(M_{9,8} = -102\). Προσεγγιστικά για το μέσο του ανοίγματος θα είναι.
\[
ql^2/12 = \left(M_{8,9} + M_{9,8}\right)/2 = \left(170 + 102\right)/2 = 136 \Rightarrow ql^2 = 1632
\]

\noindent
και

\[
M_{8,9,mid} = ql^2/24 = 1632/24 = 68 KNm
\]
