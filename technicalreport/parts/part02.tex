\pagestyle{fancy}
\chapter{Επίλυση}
\section{Διαστασιολόγηση διαμήκων οπλισμών των δοκών του 1ου ορόφου}
\noindent
Για τις δοκούς με διατομή $0.3m\times0.6m$ υπολογίζεται η επικάλυψη και το στατικό ύψος ως εξής.

\[
c_{nom} = 50 mm \rightarrow d_1 = c_{nom} + \phi_h + \phi_L/2 = 50 + 8 + 14/2 = 65 mm
\]

\noindent
και

\[
d = 600 - 65 = 535 mm
\]

\noindent
Το συνεργαζόμενο πλάτος \(b_{eff}\) στις διατομές (περιοχές) των δοκών όπου συναντάται θλίψη στο άνω μέρος προκύπτουν ως εξής.

\[
b_{eff,7} = 0.5\cdot6/10 + 0.2\cdot3 = 0.9 > 0.2\cdot0.5\cdot6 = 0.6 \Rightarrow b_{eff,8} = 0.6m
\]

\[
b_{eff,78,mid} = 0.7\cdot6/10 + 0.2\cdot3 = 1.02 > 0.2\cdot0.7\cdot6 = 0.84 \Rightarrow b_{eff,78,mid} = 0.84m
\]

\[
b_{eff,89,mid} = 0.7\cdot6/10 + 0.2\cdot3 = 1.02 > 0.2\cdot0.7\cdot6 = 0.84 \Rightarrow b_{eff,78,mid} = 0.84m
\]

\[
b_{eff,9} = 0.5\cdot6/10 + 0.2\cdot3 = 0.9 > 0.2\cdot0.5\cdot6 = 0.6 \Rightarrow b_{eff,9} = 0.6m
\]

\noindent
Για τις δοκούς και για ΚΠΜ οι κατασκευαστικοί κανόνες που ορίζονται από τους Ευρωκώδικες 2 και 8 απαιτούν τα παρακάτω, με βάση τα οποία επιλέγονται οι τοποθετούμενοι οπλισμοί. Χάριν συντομίας, οι σχετικές συγκρίσεις με τα κατασκευαστικά ελάχιστα παραλείπονται, όμως διενεργούνται σε όλους τους υπολογισμούς και τους επηρεάζουν αναλόγως.

\[
\rho_{min} = \dfrac{A_{s,min}}{bd} = 0.5\dfrac{f_{ctm}}{f_{yk}} = 0.5\cdot \dfrac{2.56}{500} = 5.12\cdot 10^{-3}
\]

\[
A_s = 308mm^2
\]

\[
A_{s,bottom,support} \geq 0.5A_{s,top,support}
\]

\[
A_{s,bottom,support} \geq 0.25A_{s,bottom,mid}
\]

\noindent
Για φόρτιση \(1.35G + 1.5Q\), από τη διαφορά του αθροίσματος των αξονικών του εισογείου και του 1ου ορόφου διαιρεμένη με $12m$ προκύπτει το φορτίο, έστω $q (KN/m)$, που αναλαμβάνουν οι δοκοί του 1ου ορόφου. Το ημιάθροισμα των ροπών των στηρίξεων κάθε δοκού είναι $-(170+102)/2= -136KNm$.

\[
q = \dfrac{\left(714+1785+714\right)-\left(510+1190+510\right)}{12} = 84 KN/m
\]

\[
M_{7,8,mid} = M_{8,9,mid} = - 136 + ql^2/8 = -136 + 84\cdot6^2/8 = 242 KNm
\]

\subsection{Μέλος 78}
\noindent
\underline{Δεξιά παρειά κόμβου 7}:

\[
\left.
   \begin{array}{ll}
     G+\psi_2 Q + Ε & \rightarrow M_{7,8} = -60+130 = 70 \\
     G+\psi_2 Q - Ε & \rightarrow M_{7,8} = -60-130 = -190 \\
     1.35G + 1.5Q     & \rightarrow M_{7,8} = -102
   \end{array}
\right\} \Rightarrow M_{7,8,top} = -190,\; M_{7,8,bottom} = 70
\]

\[
M_{7,8,top} = -190 \rightarrow \mu_{sd} = \dfrac{M_{sd}}{b d^2 f_{cd}} = \dfrac{190}{0.3\cdot0.535^2\cdot14.16\cdot10^3} = 0.156
\]

\[
\omega_{top} = 0.973\left( 1 - \sqrt{1 - \dfrac{2\mu_{sd}}{0.973}} \right) = 0.171
\]

\[
A_{s,top} = \omega_{top}b d \dfrac{f_{cd}}{f_{yd}} = 0.171\cdot 0.3 \cdot 0.535 \cdot \dfrac{14.16}{434.78} = 0.171\cdot 0.3 \cdot 0.535 \cdot \dfrac{14.16}{434.78} = 894mm^2
\]

\noindent
Άρα τοποθετούνται:

\[
A_{\tau} = 3\Phi20 = 942mm^2
\]

\[
M_{7,8,bottom} = 70 \rightarrow \mu_{sd} = \dfrac{M_{sd}}{b_{eff} d^2 f_{cd}} = \dfrac{70}{0.3\cdot0.535^2\cdot14.16\cdot10^3} = 0.057
\]

\[
\omega_{bottom} = 0.973\left( 1 - \sqrt{1 - \dfrac{2\mu_{sd}}{0.973}} \right) = 0.058
\]

\[
A_{s,bottom} = \omega_{bottom}b d \dfrac{f_{cd}}{f_{yd}} = 0.058\cdot 0.3 \cdot 0.535 \cdot \dfrac{14.16}{434.78} = 303mm^2
\]

\noindent
Άρα τοποθετούνται:

\[
A_{\tau} = 2\Phi20 = 628mm^2
\]

\noindent
\underline{Αριστερή παρειά κόμβου 8}:

\[
\left.
   \begin{array}{ll}
       G+\psi_2 Q + Ε & \rightarrow M_{8,7} = -100-123.5 = -223.5 \\
       G+\psi_2 Q - Ε & \rightarrow M_{8,7} = -100+123.5 = 23.5 \\
       1.35G + 1.5Q     & \rightarrow M_{8,7} = -170
   \end{array}
\right \} \Rightarrow M_{8,7,top} = -223.5,\; M_{8,7,bottom} = 23.5
\]

\[
M_{8,7,top} = -223.5 \rightarrow \mu_{sd} = \dfrac{M_{sd}}{b d^2 f_{cd}} = \dfrac{223.5}{0.3\cdot0.535^2\cdot14.16\cdot10^3} = 0.184
\]

\[
\omega_{top} = 0.973\left( 1 - \sqrt{1 - \dfrac{2\mu_{sd}}{0.973}} \right) = 0.206
\]

\[
A_{s,top} = \omega_{top}b d \dfrac{f_{cd}}{f_{yd}} = 0.206\cdot 0.3 \cdot 0.535 \cdot \dfrac{14.16}{434.78} = 0.206\cdot 0.3 \cdot 0.535 \cdot \dfrac{14.16}{434.78} = 1074mm^2
\]

\noindent
Άρα τοποθετούνται:

\[
A_{\tau} = 4\Phi20 = 1256mm^2
\]

\[
M_{8,7,bottom} = 23.5 \rightarrow \mu_{sd} = \dfrac{M_{sd}}{b_{eff} d^2 f_{cd}} = \dfrac{23.5}{0.3\cdot0.535^2\cdot14.16\cdot10^3} = 0.010
\]

\[
\omega_{bottom} = 0.973\left( 1 - \sqrt{1 - \dfrac{2\mu_{sd}}{0.973}} \right) = 0.010
\]

\[
A_{s,bottom} = \omega_{bottom}b_{eff} d \dfrac{f_{cd}}{f_{yd}} = 0.010\cdot 0.6 \cdot 0.535 \cdot \dfrac{14.16}{434.78} = 102mm^2
\]

\noindent
Άρα τοποθετούνται:

\[
A_{\tau} = 2\Phi20 = 628mm^2
\]

\noindent
\underline{Άνοιγμα}:

\bigskip

\[
M_{7,8,mid} = 242 \rightarrow \mu_{sd} = \dfrac{M_{sd}}{b_{eff} d^2 f_{cd}} = \dfrac{242}{0.84\cdot0.535^2\cdot14.16\cdot10^3} = 0.071
\]

\[
\omega_{mid} = 0.973\left( 1 - \sqrt{1 - \dfrac{2\mu_{sd}}{0.973}} \right) = 0.074
\]

\[
A_{s,mid} = \omega_{mid}b_{eff} d \dfrac{f_{cd}}{f_{yd}} = 0.074\cdot 0.84 \cdot 0.535 \cdot \dfrac{14.16}{434.78} = 0.074\cdot 0.84 \cdot 0.535 \cdot \dfrac{14.16}{434.78} = 1082 mm^2
\]

\noindent
Άρα τοποθετούνται:

\[
A_{\tau} = 4\Phi20 = 1256 mm^2
\]

\subsection{Μέλος 89}

\noindent
\underline{Δεξιά παρειά κόμβου 8}:

\[
\left.
   \begin{array}{ll}
       G+\psi_2 Q + Ε & \rightarrow M_{8,9} = -100+123.5 = 23.5 \\
       G+\psi_2 Q - Ε & \rightarrow M_{8,9} = -100-123.5 = -223.5 \\
       1.35G + 1.5Q     & \rightarrow M_{8,9} = -170
   \end{array}
\right \} \Rightarrow M_{8,9,top} = -223.5,\; M_{8,9,bottom} = 23.5
\]

\[
M_{8,9,top} = -223.5 \rightarrow \mu_{sd} = \dfrac{M_{sd}}{b d^2 f_{cd}} = \dfrac{223.5}{0.3\cdot0.535^2\cdot14.16\cdot10^3} = 0.184
\]

\[
\omega_{top} = 0.973\left( 1 - \sqrt{1 - \dfrac{2\mu_{sd}}{0.973}} \right) = 0.206
\]

\[
A_{s,top} = \omega_{top}b d \dfrac{f_{cd}}{f_{yd}} = 0.206\cdot 0.3 \cdot 0.535 \cdot \dfrac{14.16}{434.78} = 0.206\cdot 0.3 \cdot 0.535 \cdot \dfrac{14.16}{434.78} = 1074mm^2
\]

\noindent
Άρα τοποθετούνται:

\[
A_{\tau} = 4\Phi20 = 1256mm^2
\]

\[
M_{8,9,bottom} = 23.5 \rightarrow \mu_{sd} = \dfrac{M_{sd}}{b_{eff} d^2 f_{cd}} = \dfrac{23.5}{0.3\cdot0.535^2\cdot14.16\cdot10^3} = 0.010
\]

\[
\omega_{bottom} = 0.973\left( 1 - \sqrt{1 - \dfrac{2\mu_{sd}}{0.973}} \right) = 0.010
\]

\[
A_{s,bottom} = \omega_{bottom}b_{eff} d \dfrac{f_{cd}}{f_{yd}} = 0.010\cdot 0.6 \cdot 0.535 \cdot \dfrac{14.16}{434.78} = 102mm^2
\]

\noindent
Άρα τοποθετούνται:

\[
A_{\tau} = 2\Phi20 = 628mm^2
\]


\noindent
\underline{Αριστερή παρειά κόμβου 9}:

\[
\left.
   \begin{array}{ll}
       G+\psi_2 Q + Ε & \rightarrow M_{9,8} = -60-130 = -190 \\
       G+\psi_2 Q - Ε & \rightarrow M_{9,8} = -60+130 = 70 \\
       1.35G + 1.5Q     & \rightarrow M_{9,8} = -102
   \end{array}
\right \} \Rightarrow M_{9,8,top} = -190,\; M_{9,8,bottom} = 70
\]

\[
M_{9,8,top} = -190 \rightarrow \mu_{sd} = \dfrac{M_{sd}}{b d^2 f_{cd}} = \dfrac{190}{0.3\cdot0.535^2\cdot14.16\cdot10^3} = 0.156
\]

\[
\omega_{top} = 0.973\left( 1 - \sqrt{1 - \dfrac{2\mu_{sd}}{0.973}} \right) = 0.171
\]

\[
A_{s,top} = \omega_{top}b d \dfrac{f_{cd}}{f_{yd}} = 0.171\cdot 0.3 \cdot 0.535 \cdot \dfrac{14.16}{434.78} = 0.171\cdot 0.3 \cdot 0.535 \cdot \dfrac{14.16}{434.78} = 894mm^2
\]

\noindent
Άρα τοποθετούνται:

\[
A_{\tau} = 3\Phi20 = 942mm^2
\]

\[
M_{9,8,bottom} = 70 \rightarrow \mu_{sd} = \dfrac{M_{sd}}{b_{eff} d^2 f_{cd}} = \dfrac{70}{0.3\cdot0.535^2\cdot14.16\cdot10^3} = 0.057
\]

\[
\omega_{bottom} = 0.973\left( 1 - \sqrt{1 - \dfrac{2\mu_{sd}}{0.973}} \right) = 0.058
\]

\[
A_{s,bottom} = \omega_{bottom}b d \dfrac{f_{cd}}{f_{yd}} = 0.058\cdot 0.3 \cdot 0.535 \cdot \dfrac{14.16}{434.78} = 303mm^2
\]

\noindent
Άρα τοποθετούνται:

\[
A_{\tau} = 2\Phi20 = 628mm^2
\]

\noindent
\underline{Άνοιγμα}:

\bigskip

\[
M_{9,8,mid} = 242 \rightarrow \mu_{sd} = \dfrac{M_{sd}}{b_{eff} d^2 f_{cd}} = \dfrac{242}{0.84\cdot0.535^2\cdot14.16\cdot10^3} = 0.071
\]

\[
\omega_{mid} = 0.973\left( 1 - \sqrt{1 - \dfrac{2\mu_{sd}}{0.973}} \right) = 0.074
\]

\[
A_{s,mid} = \omega_{mid}b_{eff} d \dfrac{f_{cd}}{f_{yd}} = 0.074\cdot 0.84 \cdot 0.535 \cdot \dfrac{14.16}{434.78} = 0.074\cdot 0.84 \cdot 0.535 \cdot \dfrac{14.16}{434.78} = 1082 mm^2
\]

\noindent
Άρα τοποθετούνται:

\[
A_{\tau} = 4\Phi20 = 1256 mm^2
\]

\bigskip

\noindent\textbf{\textcolor{mygreen}{Άρα τελικά οι διαµήκεις οπλισµοί των δοκών του 1ου ορόφου, όπως προέκυψαν από τους άνωθεν υπολογισμούς, φαίνονται συγκεντρωτικά στον επόμενο πίνακα.}}

\begin{table}[h]
\centering\footnotesize
\begin{tabular}{| c || c | c | c | c | c |}
\hline
Θέση & Κόμβος 7 & Άνοιγμα 78 & Κόμβος 8 & Άνοιγμα 89 & Κόμβος 9 \\
\hline
\hline
Πάνω & $3\Phi20$ & $-$ & $4\Phi20$ & $-$ & $3\Phi20$ \\
\hline
Κάτω & $2\Phi20$ & $4\Phi20$ & $2\Phi20$ & $4\Phi20$ & $2\Phi20$ \\
\hline
\end{tabular}
\caption{Διαμήκεις οπλισμοί δοκών 1ου ορόφου}
\label{tab:beamrebar}
\end{table}

\section{Διαστασιολόγηση κατακόρυφων οπλισμών των υποστυλωμάτων του 1ου ορόφου σύμφωνα με τον ικανοτικό σχεδιασμό σε κάμψη}
