\pagestyle{fancy}
\chapter{Εισαγωγή}
\section{Ορισμός του προβλήματος}
Η παρούσα αναφορά πραγματεύεται το πρόβλημα αντισεισμικού σχεδιασμού 3-όροφου επίπεδου πλαισίου δύο ανοιγμάτων. Στις ενότητες που ακολουθούν παρουσιάζονται τα δεδομένα, όπως αυτά προκύπτουν από την ελαστική στατική ανάλυση της κατασκευής και παραμετρικοποιημένα με τον αριθμό μητρώου του συγγραφέα της εν λόγω αναφοράς, και τα ζητούμενα του προβλήματος. Στο δεύτερο μέρος γίνεται η αναλυτική επίλυση του προβλήματος και παρουσιάζονται τα αποτελέσματα αυτής.

\section{Δεδομένα}
Τα δεδομένα του προβλήματος που αφορούν τα εντατικά μεγέθη (στο καθαρό μήκος των μελών) και προέκυψαν από την ελαστική στατική ανάλυση της κατασκευής, φαίνονται στους πίνακες παρακάτω (Πίν. \ref{tab:momentbeams}, \ref{tab:momentcolumns}, \ref{tab:axialload}). Επιπλέον δεδομένα του προβλήματος είναι επιγραμματικά τα εξής.

\begin{itemize}
  \item Το ύψος ορόφου είναι $3m$ και τα ανοίγματα $6m$ (από μέσο στήριξης σε μέσο στήριξης).
  \item Το μεσαίο υποστύλωμα έχει διατομή $0.4m\times0.4m$ και τα ακραία $0.35m\times0.35m$.
  \item Οι δοκοί έχουν διατομή $0.3m\times0.6m$ και οι πλάκες έχουν πάχος $0.15m$.
  \item Κατηγορία Πλαστιμότητας Μέση κατά Ευρωκώδικα $8$ (ΚΠΜ).
  \item Σκυρόδεμα $C25/30$ και χάλυβας $B500C$.
\end{itemize}

\begin{landscape}

\begin{table}[h]
\centering\footnotesize
\begin{tabular}{| c || c | c | c | c | c | c | c | c | c | c | c | c |}
\hline
& $M_{1,2}$ & $M_{2,1}$ & $M_{2,3}$ & $M_{3,2}$ & $M_{4,5}$ & $M_{5,4}$ & $M_{5,6}$ & $M_{6,5}$ & $M_{7,8}$ & $M_{8,7}$ & $M_{8,9}$ & $M_{9,8}$ \\
\hline
\hline
$+E$ & $65$ & $-58.5$ & $58.5$ & $-65$ & $110$ & $-104.5$ & $104.5$ & $-110$ & $130$ & $-123.5$ & $123.5$ & $-130$ \\
\hline
$-E$ & $-65$ & $58.5$ & $-58.5$ & $65$ & $-110$ & $104.5$ & $-104.5$ & $110$ & $-130$ & $123.5$ & $-123.5$ & $130$ \\
\hline
$G+\psi_2 Q$ & $-50$ & $-100$ & $-100$ & $-50$ & $-70$ & $-95$ & $-95$ & $-70$ & $-60$ & $-100$ & $-100$ & $-60$ \\
\hline
$1.35G+1.5Q$ & $-85$ & $-170$ & $-170$ & $-85$ & $-119$ & $-161.5$ & $-161.5$ & $-119$ & $-102$ & $-170$ & $-170$ & $-102$ \\
\hline
\end{tabular}
\caption{Ροπές δοκών στις παρειές των κόμβων (\textlatin{KNm})}
\label{tab:momentbeams}
\end{table}

\begin{table}[h]
\centering\footnotesize
\begin{tabular}{| c || c | c | c | c | c | c | c | c | c | c | c | c | c | c | c | c | c | c |}
\hline
& $M_{1,4}$ & $M_{4,1}$ & $M_{2,5}$ & $M_{5,2}$ & $M_{3,6}$ & $M_{6,3}$ & $M_{4,7}$ & $M_{7,4}$ & $M_{5,8}$ & $M_{8,5}$ & $M_{6,9}$ & $M_{9,6}$ & $M_{7,10}$ & $M_{10,7}$ & $M_{8,11}$ & $M_{11,8}$ & $M_{9,12}$ & $M_{12,9}$ \\
\hline
\hline
$+E$ & $40$ & $-36$ & $80$ & $-72$ & $-40$ & $36$ & $65$ & $-58.5$ & $110$ & $-99$ & $-65$ & $58.5$ & $65$ & $-85$ & $110$ & $-130$ & $-65$ & $85$ \\
\hline
$-E$ & $-40$ & $36$ & $-80$ & $72$ & $40$ & $-36$ & $-65$ & $58.5$ & $-110$ & $99$ & $65$ & $-58.5$ & $-65$ & $85$ & $-110$ & $130$ & $65$ & $-85$ \\
\hline
$G+\psi_2 Q$ & $-45$ & $40$ & $0$ & $0$ & $-45$ & $40$ & $-35$ & $36$ & $0$ & $0$ & $-35$ & $36$ & $-28$ & $20$ & $0$ & $0$ & $-28$ & $20$ \\
\hline
\end{tabular}
\caption{Ροπές υποστυλωμάτων στις παρειές των κόμβων (\textlatin{KNm})}
\label{tab:momentcolumns}
\end{table}

\begin{table}[h]
\centering\footnotesize
\begin{tabular}{| c || c | c | c | c | c | c | c | c | c |}
\hline
& $N_{1,4}$ & $N_{4,7}$ & $N_{7,10}$ & $N_{2,5}$ & $N_{5,8}$ & $N_{8,11}$ & $N_{3,6}$ & $N_{6,9}$ & $N_{9,12}$ \\
\hline
\hline
$+E$ & $20$ & $60$ & $120$ & $0$ & $0$ & $0$ & $-20$ & $-60$ & $-120$ \\
\hline
$-E$ & $-20$ & $-60$ & $-120$ & $0$ & $0$ & $0$ & $20$ & $60$ & $120$ \\
\hline
$G+\psi_2 Q$ & $-140$ & $-300$ & $-420$ & $-350$ & $-700$ & $-1050$ & $-140$ & $-300$ & $-420$ \\
\hline
$1.35G+1.5Q$ & $-238$ & $-510$ & $-714$ & $-595$ & $-1190$ & $-1785$ & $-238$ & $-510$ & $-714$ \\
\hline
\end{tabular}
\caption{Αξονικές δυνάμεις μελών (\textlatin{KN})}
\label{tab:momentbeams}
\end{table}

\end{landscape}
